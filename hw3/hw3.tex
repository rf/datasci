\documentclass[10pt]{amsart}

\usepackage{color}
\usepackage{enumitem}
\usepackage{verbatim}
\usepackage{graphicx} 
\usepackage{fancyhdr}

\setlength{\headsep}{.33in}
\setlength{\topsep}{0in}
\setlength{\topmargin}{-0.2in}
\setlength{\topskip}{0in}    % between header and text
\setlength{\textheight}{9.8in} % height of main text
\setlength{\textwidth}{7.05in}    % width of text
\setlength{\oddsidemargin}{-0.30in} % odd page left margin
\setlength{\evensidemargin}{-0.30in} % even page left margin
\setlength{\parindent}{0in}   % remove paragraph indenting

\linespread{1.13}

\newcommand{\head}[1]{
   \begin{tabular*}{7.1in}{@{}l@{\extracolsep{\fill}}r}
      Russell Frank & \today \\
   \end{tabular*}
   \begin{center} \LARGE #1 \normalsize \end{center}
   \vskip 0.1in
}

\def\wl{\par \vspace{\baselineskip}}

\pagestyle{fancy}
\fancyhead[R]{rfranknj \thepage / 2}

\begin{document}

\head{Data Science HW 2}

\begin{center}

\begin{tabular}{l|l}
  1a & 2 \\
  1b & 2 \\
  2 & not done \\
  3a & 2 \\
  3b & 2 \\
  4 & 2 \\
  5 & 2 \\
  6 & not done \\
  7 & 2 \\
  8a & 2 \\
  8b & 2 \\
  9 & 2 \\
  10 & 2 \\
  11 & 2 \\
\end{tabular}

\end{center}

\newpage

\textbf{7a.} Well, we can graph the average distance to centroid versus the 
number of clusters $k$; then, we choose the value of $k$ corresponding to 
where the average distance to centroid begins to drop slowly. In other words,
at some point increasing the number of clusters stops significantly decreasing
the average distance to a centroid, and this would be a good value of $k$ to
choose. \\

\textbf{7b.} As described previously, if we're choosing a value for $k$ that
seems to produce a small average distance to centroid, it should be a good 
value for the number of clusters. \\

\textbf{8.} The simplest way would be to choose the clustering with the lowest
total SSE. \\

\textbf{9a.} 36 \\
%Centroid: (3, 5). SSE: 
%10.2426406871

\textbf{9b.} c1: ((3, 2), (6, 5))   c2: ((0, 8)) \\

\textbf{9c.} The new error is 10, so the reduction is 26. \\


\end{document}
